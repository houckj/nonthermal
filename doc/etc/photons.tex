\documentclass[12pt]{article}

\usepackage{amsmath}
\usepackage{psfig}
\usepackage{bm}

\evensidemargin=1truein
\oddsidemargin=1truein
\hoffset=-1truein
\voffset=-1truecm
\textheight=9.5truein
\textwidth=6.5truein

\newcommand{\D}{{\rm d}}

\begin{document}

\begin{center}
{\Large\bf
Computation of Selected Nonthermal Photon Spectra}\\
\large
John C. Houck \\
\verb|<houck@space.mit.edu>|
\end{center}

\section{Synchrotron Radiation}

From Blumenthal \& Gould (1970),
the total synchrotron power emitted per unit frequency
from an electron spiraling in a magnetic field is
\begin{equation}
P_{\rm emitted}(\nu) =
\frac{\sqrt{3} e^3 B \sin\alpha}{mc^2}
\frac{\nu}{\nu_c} \int_{\nu/\nu_c}^\infty \D\xi K_{5/3}(\xi)
\label{eq:p-emitted}
\end{equation}
where $B$ is the magnetic field strength, and
$\alpha$ is the pitch angle between the electron's velocity
vector ${\bf v}$ and the magnetic field vector ${\bf B}$.
In (\ref{eq:p-emitted}), $\nu_c$ is the critical frequency, defined as
\begin{equation}
  \nu_c = \frac{3eB\gamma^2}{4\pi m c} \sin\alpha
  \equiv \nu_0 \gamma^2 \sin\alpha,
\label{eq:nu-c}
\end{equation}

We assume a time-independent electron distribution and define
$N(\gamma, \alpha, {\bf r})~d\gamma~d\Omega_\alpha$, as the
density of nonthermal electrons at ${\bf r}$, with pitch angle
$\alpha$ within $d\Omega_\alpha$ and energy $\gamma$ within
$d\gamma$. From Blumenthal \& Gould (1970), the total
synchrotron power received per unit volume per unit frequency
averaged over pitch-angles is
\begin{equation}
\frac{dW}{d\nu dt} = \frac{1}{4\pi}\int \D\gamma \int \D\Omega_\alpha~
              P_{\rm emitted}(\nu)~N(\gamma, \alpha, {\bf r})
\label{eq:dwdnudt}
\end{equation}

Combining (\ref{eq:p-emitted}), (\ref{eq:nu-c}) and
(\ref{eq:dwdnudt}), and assuming an isotropic distribution
of pitch-angles, we obtain
\begin{equation}
\frac{dW}{d\nu dt} =
\frac{\sqrt{3} e^3 B}{4\pi mc^2}
\int \D\gamma N(\gamma) \int \D\phi_\alpha~ \int \D\alpha
\sin^2\alpha ~F\left(\frac{\nu}{\nu_0\gamma^2\sin\alpha}\right).
\end{equation}
Substituting $t\equiv \sin\alpha$, and
expressing the result in terms of the photon emission rate, we have
\begin{equation}
\varepsilon\frac{dn_\varepsilon}{d\varepsilon dt} =
\frac{\sqrt{3} e^3 B}{hmc^2}
\int_0^\infty \D\gamma N(\gamma) \int_0^1 \D t
\frac{t^2}{\sqrt{1 - t^2}} F\left(\frac{\nu}{\nu_0\gamma^2 t}\right),
\label{eq:sync-emis1}
\end{equation}
where $\varepsilon = h\nu/(mc^2)$ is the photon energy in units
of the electron rest energy.

Writing the inner integral in the form
\begin{equation}
I(x) = \int_0^1 \D t \frac{t^2 F(x/t)}{\sqrt{1 - t^2}}
\end{equation}
it is clearly a function of a single variable and may be
precomputed to speed numerical evaluation of (\ref{eq:sync-emis1})
(see the upper panel of Fig \ref{anglefun-fig}).
The asymptotic behavior of $F(x)$ ensures that the integrand
is well behaved at the limits of integration:
\begin{align}
F(\xi) &=
\frac{4\pi}{\sqrt{3}\Gamma(1/3)}\left(\frac{\xi}{2}\right)^{1/3}
     & \xi \ll 1 \\
F(\xi) &=
\left(\frac{\pi}{2}\right)^{1/2}\xi^{1/2} \exp(-\xi)
     & \xi \gg 1
\end{align}
The lower panel Fig \ref{anglefun-fig} shows
effect of angle-averaging over an isotropic distribution of pitch
angles.  For $t \ll 1$, $I(t) \approx 0.84131 F(x)$.

Expressing (\ref{eq:sync-emis1}) in terms of $I(x)$, we have
\begin{equation}
\frac{dn_\varepsilon}{d\varepsilon dt} =
\frac{\sqrt{3} e^3 B}{hmc^2\varepsilon}
\int_0^\infty \D\gamma N(\gamma) I\left(\frac{\varepsilon}{\varepsilon_0\gamma^2}\right),
\label{eq:sync-emis}
\end{equation}

\begin{figure}
\hbox{\hspace*{-0.25in}\psfig{file=angfun.ps,width=5.25in,angle=-90}}
\caption{}
\label{anglefun-fig}
\end{figure}

\clearpage
\newpage

\section{Inverse Compton Scattering}

Given a distribution of nonthermal electrons $N(\gamma)$, immersed
in an isotropic radiation field with photon number density
$n(\varepsilon_i)$, the total spectrum of Compton scattered photons
is
\begin{equation}
\frac{dn_\varepsilon}{d\varepsilon dt} =
c \int \D\gamma N(\gamma) \int \D\varepsilon_i
  n(\varepsilon_i)
  \sigma_{\rm K-N}(\gamma, \varepsilon_i, \varepsilon).
\end{equation}
In this expression,
\begin{equation}
 \sigma_{\rm K-N}(\gamma, \varepsilon_i, \varepsilon) =
\frac{2\pi r_0^2}{\varepsilon_i \gamma^2}
\left[
1 + q - 2q^2 + 2q\ln q +
\frac{\Gamma^2 q^2 (1-q)}{2(1 + \Gamma q)}
\right]
\end{equation}
is the Klein-Nishina scattering cross-section,
where $\Gamma \equiv 4\varepsilon_i\gamma$ and
\begin{equation}
 q \equiv \frac{\varepsilon}
             {4 \varepsilon_i \gamma (\gamma - \varepsilon)},
             (0 \le q \le 1).
\end{equation}
The constant $r_0 = e^2 /(m_ec^2)$ is the classial electron
radius ($\sigma_T = 8\pi r_0^2/3$ is the Thomson cross-section).
For a blackbody radiation field,
the photon number density is
\begin{equation}
  n(\varepsilon) = \frac{1}{\pi^2 \lambda^3}
  \frac{\varepsilon^2}{e^{\varepsilon/\Theta} - 1}
\end{equation}
where $\varepsilon \equiv h\nu/(mc^2)$, $\lambda \equiv \hbar/(mc)$
is the electron Compton wavelength
and $\Theta \equiv kT/(mc^2)$.

It is useful to tabulate the integral over $\varepsilon_i$
for a given spectrum of seed photons, $n(\varepsilon_i)$.
The resulting table gives the value of the integral as a function
of $\gamma$ and $\varepsilon$.
For the cosmic background radiation field, the resulting table
is shown in Fig \ref{ic-fig}.

\begin{figure}
\hbox{\hspace*{-0.25in}\psfig{file=ic_table_CBR.ps,width=5.25in,angle=-90}}
\caption{}
\label{ic-fig}
\end{figure}

\clearpage
\newpage

\section{Nonthermal $e^{-}e^{-}$ Bremsstrahlung}

The rate of interactions between incident and target particles
per unit space-time volume may be in written in covariant form as
\begin{equation}
  \frac{d^4N}{dx^4} = \left({\tilde j_1}\cdot{\tilde j_2}\right) \beta_r \sigma
\end{equation}
where ${\tilde j_i} = n_i(1,{\bm \beta_i})$ is the particle-current
4-vector for the $i$th species, with density $n_i$ and velocity
3-vector ${\bm \beta_i}$.  In this expression, $\beta_r$ is the invariant
relative velocity
\begin{equation}
  \beta_r^2 = \frac{\left({\tilde p_1}\cdot{\tilde p_2}\right)^2
              - m_1^2 m_2^2}{\left({\tilde p_1}\cdot{\tilde p_2}\right)^2}
\end{equation}
where ${\tilde p_i} = m_i{\tilde u_i}$ is the 4-momentum
and $m_i$ is the particle mass.  The total cross-section for
the interaction is $\sigma$ (and is {\it required} to be invariant).

Following Dermer (1984, ApJ 280, 328), the relativistic
reaction rate $r$ is defined as the number of interactions per
unit space-time volume and is also an invariant quantity.
The reaction rate for two interacting distributions of
identical particles is given by
\begin{equation}
 r = \int\cdots\int \left(1 - {\bm \beta_1} \cdot {\bm \beta_2}\right) \beta_r d\sigma dn_1 dn_2
\end{equation}
where the integrals extend over all momenta of the interacting
distributions.

Given azimuthal symmetry, the particle collisions may be
characterized by three variables, the two particle momenta and
the cosine of the angle between them. Writing the reaction rate
integral in terms of particle distributions, $f_i(p_i)$,
the lab-frame emitted photon spectrum is
\begin{equation}
 S(k) = \left(2\pi\right)^2
        \int dp_1 p_1^2 f_1(p_1)
        \int dp_2 p_2^2 f_2(p_2)
        \int d(\cos\theta)
        \left(1 - {\bm \beta_1} \cdot {\bm \beta_2}\right)
        \beta_r \frac{d\sigma}{dk}
\end{equation}
where $\theta$ is the angle between the two particle momenta
and $d\sigma/dk$ is the differential cross-section for emitting
a photon of energy $k$.

Specializing to the case where $\beta_1 \gg \beta_2$, the
target particles are essentially stationary so that ${\bm
\beta_1} \cdot {\bm \beta_2} = 0$. Replacing $f_2(p_2)$ with a
delta function, $\delta(p_2)$, the integral over target momenta
can be evaluated trivially. The lab-frame photon spectrum is
then
\begin{equation}
S(k) = n_2 \int_{1 + k/m_1}^\infty d\gamma_1
        f_1^\prime(\gamma_1) \beta_r \frac{d\sigma}{dk}.
\label{eebrems-spectrum}
\end{equation}
where we have replaced the integral over momentum space
with the corresponding integral over incident electron energies.
The integral limits span the range of electron energies
which can produce a photon of energy $k$.

Because the theoretical cross-section is derived in the center
of momentum (CM) frame, we will derive the lab-frame
differential cross-section by applying a Lorentz transformation
to the CM frame result.  We will make use of the Lorentz
invariant
\begin{equation}
   k \frac{d^3\sigma}{dp^3}
\end{equation}
where ${\bm p}$ is the momentum 3-vector.
To demonstrate that this quantity is a Lorentz invariant,
consider components of the momentum parallel and perpendicular
to the direction of motion,
\begin{equation}
   k \frac{d^3\sigma}{dp^3} = k \frac{d^3\sigma}{dp_{||}d^2p_\perp}
\end{equation}
and note that under a Lorentz transformation,
$k$ and $dp_{||}$ behave the same way while the perpendicular
components are unchanged.

Equating the lab frame and CM frame values
of this invariant, we have
\begin{equation}
 k \frac{d^3\sigma}{p^2 dp d\Omega} =
  k_* \frac{d^3\sigma_*}{p_*^2 dp_* d\Omega_*}
\end{equation}
where starred quantities are to be evaluated in the CM frame.
The photon energy-momentum 4-vector is $k(1,{\bm n})$, where
${\bm n}$ is a unit 3-vector in the propagation direction.  It
follows that $p^2 dp = k^2 dk$ so that, integrating over
lab-frame solid angles and assuming azimuthal symmetry, we have
\begin{equation}
  \frac{d\sigma}{dk} =
   2\pi \int_{\mu_{\rm min}}^1 d\mu \frac{k}{k_*}
  \frac{d^3\sigma_*}{dk_* d\Omega_*}
\label{sigma-angular-integral}
\end{equation}
where, $\mu \equiv \cos\psi$ and the integral extends over all lab-frame angles
which are accessible to emitted photons.  In this expression,
the differential cross-section is to be evaluated at
the appropriately Lorentz-transformed values of the electron
energy, photon energy and photon emission angle:
\begin{equation}
\frac{d^3\sigma_*}{dk_* d\Omega_*} \equiv
F_*\left[k_*(\gamma, k, \psi), \gamma_*(\gamma, k, \psi), \psi_*(\gamma, k, \psi)\right].
\end{equation}
The differential cross-section $F_*(\gamma_*, k_*, \psi_*)$,
having units of e.g. barns~keV${}^{-1}$~sr${}^{-1}$, is computed using
software kindly provided by Eberhard Haug.

To determine the Lorentz transformation between the lab
frame and the CM frame, consider the Lorentz invariant
product, $\eta \equiv {\tilde p_1}\cdot{\tilde p_1}$.
In the CM frame, $|{\bm \beta_c}| \equiv {\bm \beta_1} = -{\bm \beta_2}$, so that
$\eta_{\rm cm} = \gamma_c^2 (1+\beta_c^2)$.  In the lab frame,
particle 2 is arbitrarily chosen to be at rest so that
$\eta_{\rm lab} = \gamma_1$.  Requiring $\eta_{\rm lab} =
\eta_{\rm cm}$ yields
\begin{equation}
  \gamma_c \equiv \frac{1}{\sqrt{1 - \beta_c^2}} =
\left(\frac{\gamma_1 + 1}{2}\right)^{1/2},
\end{equation}
which determines the desired Lorentz transformation.

In the lab frame, the emitted photon has energy $k$ and is
emitted with direction angle $\psi$ so that the
CM frame photon energy is
\begin{equation}
k_* = k \gamma_c \left(1 - \beta_c \cos\psi\right)
\end{equation}
and the CM frame photon direction cosine is
\begin{equation}
\cos\psi_* = \frac{\cos\psi - \beta_c}{1 - \beta_c\cos\psi}.
\end{equation}
In general, the electron energy
in a moving frame is
\begin{equation}
 E_* = \gamma_c \left(E - \beta_c p \cos\omega\right)
\end{equation}
where the electron energy-momentum 4-vector is
$m_1\gamma_1 (1, \beta_1)$, and $\omega$ is the angle between the velocity 3-vectors
of the electron and of the moving frame.  Because the
electron velocity is, by definition, parallel to the
CM frame velocity vector, $\omega \equiv 0$ so that,
for the electron,
\begin{equation}
 \gamma_* = \gamma_c \gamma_1 \left(1 - \beta_c \beta_1\right).
\end{equation}

The limits on the lab-frame angular integral over the
cross-section (\ref{sigma-angular-integral}) are given by Haug
(1975) and follow from energy-momentum conservation.
In the CM frame, the photon emission is symmetric along
the direction of motion and all values of $\psi_*$ are
accessible.  In the lab frame, beaming restricts the photon
emission angles to a narrow cone in the forward direction.
From Haug (1975), the limits of integration are
${\psi_{\rm min}} = 0$ and, for photons with energies
$(\gamma_1 -1)/(\gamma_1 + \gamma_1 \beta_1 + 1) < k \leq k_{\rm
max} \equiv (\gamma_1 -1)/(\gamma_1 - \gamma_1 \beta_1 + 1)$,
the maximum emission angle, $\psi_{\rm max}$, is determined by
\begin{equation}
\cos {\psi_{\rm max}} = \frac{(\gamma_1 + 1)k - (\gamma_1-1)}{k \gamma_1 \beta_1}
\end{equation}
Lower energy photons may span $\pi < \psi < 0$.

In principle, the photon spectrum due to nonthermal $e^-e^-$
bremsstrahlung is computed using the differential cross-section
from (\ref{sigma-angular-integral}) in
(\ref{eebrems-spectrum}). However, because of relativistic
beaming, numerical evaluation of the integral in
(\ref{sigma-angular-integral}) can become dominated by
round-off errors associated with finite-precision arithmetic.
The problem is that the factors $(\cos \psi - \beta_c)$ and $(1
- \beta_c \cos\psi)$ and $\gamma_c\gamma_1(1 - \beta_c
\beta_1)$ are dominated by round-off errors in the limit that
$\beta_c \rightarrow 1$, $\beta_1 \rightarrow 1$ and $\psi
\rightarrow 0$.

To avoid this loss of numerical precision, we isolate the
neighborhood of $\psi = 0$ by
breaking (\ref{sigma-angular-integral}) into two pieces,
\begin{equation}
  \frac{1}{2\pi}\frac{d\sigma}{dk} \equiv I_1 + I_\epsilon
  \equiv
   \int_{\mu_{\rm min}}^{1 - \epsilon} d\mu f(\mu)
   + \int_{1-\epsilon}^1 d\mu f(\mu),
\end{equation}
where $f(\mu) \equiv (k/k_*)F_*(\gamma_*, k_*, \psi_*)$.
$I_1$ can be evaluated as it stands because $\epsilon > 0$ is
chosen large enough to avoid the difficult region. If $\mu_{\rm min}
\geq 1 - \epsilon$, $I_1 \equiv 0$ and the entire contribution
comes from $I_\epsilon$. To evalute $I_\epsilon$, we introduce
a change of variables to ensure that the numerical cancellation
is handled analytically, retaining full machine precision
as much as possible. Defining $\mu \equiv 1 - u$,
the second integral may be written
\begin{equation}
  I_\epsilon = \int_0^\epsilon du f(1-u).
\end{equation}
In this form, the integrand may be computed directly in terms of
numerically small values.  To do that, we rewrite the
Lorentz transform expressions in terms of $\mu \equiv 1-u$,
$\beta \equiv 1 - \delta\beta$, and
$\beta_c \equiv 1 - \delta\beta_c$.  The problematic factors
then become
\begin{eqnarray}
 &\mu - \beta = \delta\beta - \delta\mu \nonumber\\
 &1 - \beta\mu = \delta\mu + \delta\beta - \delta\mu\delta\beta \\
 &\gamma_c \gamma_1 (1 - \beta_c\beta_1) =
 \frac{\delta\beta_c + \delta\beta_1 - \delta\beta_c\delta\beta_1}{
\sqrt{\delta\beta_1\delta\beta_c(2-\delta\beta_1)(2-\delta\beta_c)}\nonumber
 }
\end{eqnarray}
We then compute $\delta\beta_1$ and $\delta\beta_c$
using
\begin{equation}
 \delta\beta(\gamma) \equiv 1 - \beta(\gamma) =
   \frac{1}{2}\gamma^{-2}
 + \frac{1\cdot 1}{2\cdot 4}\gamma^{-4}
 + \frac{1\cdot 1 \cdot 3}{2\cdot 4 \cdot 6}\gamma^{-6}
 + \frac{1\cdot 1 \cdot 3 \cdot 5}{2\cdot 4 \cdot 6 \cdot 8}\gamma^{-8}
 + \dots
\label{dbeta-series}
\end{equation}
The series
(\ref{dbeta-series}) converges
rapidly for $\gamma \gg 1$ so the first few terms are sufficient to achieve
high accuracy.

\end{document}
